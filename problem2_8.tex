\subsection{}
The energy flux distribution of the CR is given by
\begin{equation}
  \frac{d\Phi}{dE} = A E^{-2.7}.
\end{equation}

There are
\begin{equation}
  \int_{1\TeV}^{5\TeV} A E^{-2.7} dE \bigg/ \int_{5\TeV}^{\infty} A E^{-2.7} dE
  = \frac{1^{-1.7}-5^{-1.7}}{5^{-1.7}}
  = \boxed{ 14.4 }
\end{equation}
times as many CR between $1\TeV$ and $5\TeV$ as there are above $5\TeV$.

There are
\begin{equation}
  \frac{0.1^{-1.7}-1^{-1.7}}{5^{-1.7}}
  = \boxed{ 758 }
\end{equation}
times as many CR between $0.1\TeV$ and $1\TeV$ as there are above $5\TeV$.

\subsection{}
Let $P(x>a|\left<x\right>=b)$ denote the probability of measuring $x$ to be greater than $a$ given that the true value of x is $b$. Then,
\begin{equation}
  P(x>a|\left<x\right>=b)
  = \int_a^\infty \frac{1}{\sqrt{2\pi}\,\sigma}\ e^{-\frac{(x-b)^2}{2\sigma^2}} dx
  = \frac{1}{2} \erfc\left(\frac{a-b}{\sqrt{2}\, \sigma}\right)
\end{equation}

Let $x = \log_{10}(E/\mathrm{eV})$.
The $x$ flux distribution is
\begin{equation}
  \frac{d\Phi}{dx} = \frac{d\Phi}{dE} \frac{dE}{dx}.
\end{equation}
\begin{equation}
  E = 10^x
  \quad \Rightarrow \quad
  \frac{dE}{dx} = \ln(10)\, 10^x
  \quad \Rightarrow \quad
  \frac{d\Phi}{dx} = A \left(10^x\right)^{-2.7} \ln(10)\, 10^x
  = A \ln(10)\, 10^{-1.7x} = A' e^{-3.914 x}.
\end{equation}

The probability of measuring $x$ to be greater than $a$ given that the true value of $x$ is greater than $b$ is
\begin{equation}
  P\big(x>a|\left<x\right>>b\big)
  = \int_b^\infty \frac{1}{2}\erfc\left(\frac{a-x}{\sqrt{2}\,\sigma}\right) e^{-3.914 x}\, dx
  \bigg/ \int_b^\infty e^{-3.914 x}\, dx
\end{equation}

By Bayes' theorem,
\begin{equation}
  P(A|B) = \frac{P(B|A) P(A)}{\sum_i P(B|A_i) P(A_i)}.
\end{equation}

\begin{equation}
  P\big(\left<x\right>>5\TeV|x>5\TeV\big)
  = \frac{
    P\big(x>5\TeV|\left<x\right>>5\TeV\big) P(\left<x\right>>5\TeV)
  }{
    \int_{1\TeV}^\infty P\big(x>5\TeV|\left<x\right>>b\big) P(\left<x\right>>b)\, db
  }
\end{equation}



